\documentclass{book}

\title{Gotujemy z Martą i Tomaszem}
\date{2018-11-28}
\author{
	Marta Klimek
	\and
	Tomasz Klimek
}

\usepackage[T1]{fontenc}
\usepackage[utf8]{inputenc}
\usepackage[polish]{babel}
\usepackage{hyperref}

\renewcommand{\chaptername}{Rozdział}
\renewcommand{\contentsname}{Spis treści}

\newcommand{\skladniki}[1]{\subsection*{Składniki}
	\begin{tabular}{ll}
		#1
	\end{tabular}
}
\newcommand{\skladnik}[3]{#3&#1 #2\\}
\newcommand{\sztuk}[2]{\skladnik{#1}{szt}{#2}}
\newcommand{\gram}[2]{\skladnik{#1}{g}{#2}}
\newcommand{\litr}[2]{\skladnik{#1}{l}{#2}}
\newcommand{\lyzka}[2]{\skladnik{#1}{łyż}{#2}}
\newcommand{\lyzeczka}[2]{\skladnik{#1}{m.łyż}{#2}}
\newcommand{\przyprawa}[1]{\skladnik{}{}{#1}}

\newcommand{\przygotowanie}[1]{\subsection*{Sposób przygotowania}
	\begin{enumerate}
		#1
	\end{enumerate}
}
\newcommand{\krok}[1]{\item #1}

\newcommand{\podanie}[1]{\subsection*{Propozycja podania}#1}

\newcommand{\wartosci}[4]{\subsection*{Wartości odżywcze w 100 g}
	KCal: #1\\
	Białko: #2\\
	Tłuszcz: #3\\
	Węglowodany: #4
}

\newcommand{\zrodlo}[1]{\subsection*{Źródło}#1}
\newcommand{\link}[1]{Strona internetowa: \url{#1}}
\newcommand{\ksiazka}[3]{Ksiazka: \textbf{#1}, #2 (str.#3)}

\begin{document}
	\maketitle
	\tableofcontents

	\chapter{Zupy}
		\newpage
		
		\section{Zupa dyniowa}
			\skladniki{
				\lyzka{1}{Olej roślinny}
				\sztuk{1/2}{Cebula}
				\sztuk{1}{Pomidor}
				\sztuk{1}{Ząbek czosnku}
				\sztuk{2}{Starty imbir}
				\sztuk{1}{Mała papryczka chili}
				\sztuk{2}{Mniejsze lub średnie jabłko (zwykłe, deserowe)}
				\gram{500}{Upieczona dynia bez skóry (ewentualnie dyni świeżej)}
				\lyzeczka{1/3}{Cynamon}
				\litr{ok. 1,25}{Bulion warzywny lub wegański}
				\przyprawa{Sól}
				\przyprawa{Pieprz}
			}
			\przygotowanie{
				\krok{Podgrzać olej w większym garnku z grubym dnem, dodać cebulę i zeszklić.}
				\krok{Dodać starty czosnek, starty imbir i przekrojoną na pół, wypestkowaną chili. Smażyć przez minutę.}
				\krok{Dodać obrane, pozbawione gniazd nasiennych i pokrojone w kostkę jabłka (oraz 	surową dynię pokrojoną w kosteczkę - jeśli jej używamy). Smażyć mieszając co jakiś czas przez około 4 - 5 minut. Gdyby warzywa zaczęły przywierać do dna garnka, można dodać 	2 - 3 łyżki bulionu.}
				\krok{Dodać dynię upieczoną, pokrojoną na kawałki. Doprawić solą, pieprzem, cynamonem i zagotować.}
				\krok{Wlać bulion w takiej ilości aby sięgał 2 cm ponad poziom warzyw. Zagotować, 	przykryć i gotować przez około 10 minut na mniejszym ogniu aż dynia będzie miękka.}
				\krok{Zmiksować na gładkie puree dodając w razie potrzeby więcej zagotowanego bulionu.}
			}
			\podanie{
				\paragraph{}Świeżo zmielony kolorowy pieprz, orzechy włoskie podprażone na suchej patelni, chleb pieczony w żaroodpornym garnku z grubą i chrupiącą skórką np. z dodatkiem gorgonzoli (z TEGO przepisu). Gorącą zupę można posypać posiekaną gorgonzolą lub skropić mlekiem kokosowym.
			}
			\zrodlo{\link{https://www.kwestiasmaku.com/zielony_srodek/dynia/zupa_dyniowo_jablkowa/przepis.html}}
			\newpage
	
		\section{Tytuł przepisu} %--------------------------------------PRZYKŁAD
			\skladniki{
				\lyzka{1}{Składnik A}
				\sztuk{2}{Składnik B}
				\gram{100}{Składnik C}
				\lyzeczka{1}{Składnik D}
				\litr{3}{Składnik E}
				\przyprawa{Sól}
				\przyprawa{Pieprz}
			}
			\przygotowanie{
				\krok{Pierwszy krok}
				\krok{Drugi krok}
				\krok{Trzeci krok}
			}
			\podanie{
				\paragraph{}Jak podawać.
			}
			\wartosci{999}{99}{99}{99}
			\zrodlo{\ksiazka{Tytuł książki}{Autor książki}{123}}
			\newpage
		
	\chapter{Dania wegetariańskie}
		\newpage
	
		\section{Tytuł przepisu} %--------------------------------------PRZYKŁAD
			\skladniki{
				\lyzka{1}{Składnik A}
				\sztuk{2}{Składnik B}
				\gram{100}{Składnik C}
				\lyzeczka{1}{Składnik D}
				\litr{3}{Składnik E}
				\przyprawa{Sól}
				\przyprawa{Pieprz}
			}
			\przygotowanie{
				\krok{Pierwszy krok}
				\krok{Drugi krok}
				\krok{Trzeci krok}
			}
			\podanie{
				\paragraph{}Jak podawać.
			}
			\wartosci{999}{99}{99}{99}
			\zrodlo{\ksiazka{Tytuł książki}{Autor książki}{123}}
			\newpage
	
	\chapter{Dania rybne}
		\newpage
	
		\section{Tytuł przepisu} %--------------------------------------PRZYKŁAD
			\skladniki{
				\lyzka{1}{Składnik A}
				\sztuk{2}{Składnik B}
				\gram{100}{Składnik C}
				\lyzeczka{1}{Składnik D}
				\litr{3}{Składnik E}
				\przyprawa{Sól}
				\przyprawa{Pieprz}
			}
			\przygotowanie{
				\krok{Pierwszy krok}
				\krok{Drugi krok}
				\krok{Trzeci krok}
			}
			\podanie{
				\paragraph{}Jak podawać.
			}
			\wartosci{999}{99}{99}{99}
			\zrodlo{\ksiazka{Tytuł książki}{Autor książki}{123}}
			\newpage
	
	\chapter{Dania mięsne}
		\newpage
	
		\section{Tytuł przepisu} %--------------------------------------PRZYKŁAD
			\skladniki{
				\lyzka{1}{Składnik A}
				\sztuk{2}{Składnik B}
				\gram{100}{Składnik C}
				\lyzeczka{1}{Składnik D}
				\litr{3}{Składnik E}
				\przyprawa{Sól}
				\przyprawa{Pieprz}
			}
			\przygotowanie{
				\krok{Pierwszy krok}
				\krok{Drugi krok}
				\krok{Trzeci krok}
			}
			\podanie{
				\paragraph{}Jak podawać.
			}
			\wartosci{999}{99}{99}{99}
			\zrodlo{\ksiazka{Tytuł książki}{Autor książki}{123}}
			\newpage
	
	\chapter{Sałatki}
		\newpage
	
		\section{Tytuł przepisu} %--------------------------------------PRZYKŁAD
			\skladniki{
				\lyzka{1}{Składnik A}
				\sztuk{2}{Składnik B}
				\gram{100}{Składnik C}
				\lyzeczka{1}{Składnik D}
				\litr{3}{Składnik E}
				\przyprawa{Sól}
				\przyprawa{Pieprz}
			}
			\przygotowanie{
				\krok{Pierwszy krok}
				\krok{Drugi krok}
				\krok{Trzeci krok}
			}
			\podanie{
				\paragraph{}Jak podawać.
			}
			\wartosci{999}{99}{99}{99}
			\zrodlo{\ksiazka{Tytuł książki}{Autor książki}{123}}
			\newpage
	
	\chapter{Desery}
		\newpage
	
		\section{Tytuł przepisu} %--------------------------------------PRZYKŁAD
			\skladniki{
				\lyzka{1}{Składnik A}
				\sztuk{2}{Składnik B}
				\gram{100}{Składnik C}
				\lyzeczka{1}{Składnik D}
				\litr{3}{Składnik E}
				\przyprawa{Sól}
				\przyprawa{Pieprz}
			}
			\przygotowanie{
				\krok{Pierwszy krok}
				\krok{Drugi krok}
				\krok{Trzeci krok}
			}
			\podanie{
				\paragraph{}Jak podawać.
			}
			\wartosci{999}{99}{99}{99}
			\zrodlo{\ksiazka{Tytuł książki}{Autor książki}{123}}
			\newpage
	
	\chapter{Przetwory}
		\newpage
	
		\section{Tytuł przepisu} %--------------------------------------PRZYKŁAD
			\skladniki{
				\lyzka{1}{Składnik A}
				\sztuk{2}{Składnik B}
				\gram{100}{Składnik C}
				\lyzeczka{1}{Składnik D}
				\litr{3}{Składnik E}
				\przyprawa{Sól}
				\przyprawa{Pieprz}
			}
			\przygotowanie{
				\krok{Pierwszy krok}
				\krok{Drugi krok}
				\krok{Trzeci krok}
			}
			\podanie{
				\paragraph{}Jak podawać.
			}
			\wartosci{999}{99}{99}{99}
			\zrodlo{\ksiazka{Tytuł książki}{Autor książki}{123}}
			\newpage
	
	\chapter{Porady}
		\newpage
	
		\section{Jak upiec dynię}
			\paragraph{}Około 2 kg dyni umyć i pokroić na około 6 części, usunąć pestki. Położyć na blaszce skórką do dołu i wstawić do piekarnika nagrzanego do 200 stopni C. Piec przez około 1 godzinę, lub do czasu aż dynia będzie miękka (wbity widelec łatwo będzie zagłębiał się w miąższu). Łyżką wydrążyć miąższ z upieczonej dyni i zmiksować ją blenderem na gładkie purée. Odmierzyć potrzebną ilość.
	
		\section{Tytuł porady} %--------------------------------------PRZYKŁAD
			\paragraph{}Treść porady

\end{document}